\documentclass[10pt]{beamer}

\usetheme{sthlm}

%%%% STHLM
\usepackage{metalogo}
%\usepackage[utf8]{inputenc}
\usepackage{%
	booktabs,
	datetime,
	pgfplots,
	ragged2e,
	tabularx,
	wasysym
}
\usepackage[scale=2]{ccicons}
%%%% /STHLM


\usepackage{nameref}
\makeatletter
\newcommand*{\currentname}{\currentlabelname}
\makeatother

\newcommand*{\footnotedef}[2]{#1\footnote{\textbf{#1}: #2}}
\newcommand*{\red}[1]{{\color{red}#1}}
\newcommand*{\TODO}{\red{TODO}}
\usepackage{xcolor}

\hypersetup{%
	colorlinks,
	linkcolor=sthlmBlue,
	urlcolor=sthlmBlue,
	linkbordercolor=sthlmBlue,
	pdfborderstyle={/S/U/W 1}
%	linkcolor={red!50!black}
	%linkcolor={196!0!100},
}

% so we don't have 'Figure x.x' on captions
\setbeamertemplate{caption}{\raggedright\insertcaption\par}

\title{An Introduction to Git -- Workshop}
\subtitle{Version Control for the Greater Good}
\date{Thursday 6\textsuperscript{th} October, 2016}
\author{Jamie Tanna, \url{http://jvt.me}}
\institute{\href{http://hacksocnotts.co.uk}{Hacksoc Nottingham}
}

\usepackage{listings}
\lstset{%
	language=bash,
	keepspaces=true,
	basicstyle=\ttfamily
}

\begin{document}

\maketitle

\begin{frame}
	\frametitle{Overview}
	\setbeamertemplate{section in toc}[sections numbered]
	\tableofcontents
	% \tableofcontents[hideallsubsections]
\end{frame}

\begin{frame}{Schedule}
	\begin{table}[]
		\begin{tabularx}{\linewidth}{l>{\raggedright}X}
		\toprule
		\textbf{Event}			& \textbf{Duration} \tabularnewline
		\midrule
			Introduction to Git  Talk & 45 mins \tabularnewline
			Food & 45 mins \tabularnewline
			Hacktoberfest Workshop & 60+ mins \tabularnewline
		\bottomrule
		\end{tabularx}
	\end{table}
\end{frame}

\section{An Introduction to Git}

\begin{frame}
	\frametitle{A Few Notes}

	\begin{alertblock}{About this workshop}
			\begin{itemize}
				\item This workshop is to be interactive -- if you have any questions, let me know
			\end{itemize}
	\end{alertblock}
\end{frame}

\begin{frame}
	\frametitle{A Few Notes}

	\begin{alertblock}{Git vs Github}
		\href{http://www.git-scm.com/}{Git} $!==$ \href{https://github.com}{Github}
			\begin{itemize}
				\item Github is one of many places for git hosting -- \hyperlink{hosted_vcs_compare}{we'll compare later}
				\item A lot of people \textit{wrongly} use them interchangeably, but they are not synonymous!
			\end{itemize}
	\end{alertblock}
\end{frame}

\begin{frame}
	\frametitle{What's a Version Control System and Why Should I Use One?}

	\begin{figure}[c]
		\includegraphics[width=0.6\textwidth]{images/pre-vcs}
		\\
		Does this look familiar?
	\end{figure}

	\begin{itemize}
		\item Change management
		\item Easy collaboration
		\begin{itemize}
			\item Everyone can easily work on the same codebase without any issues
			\item Even when editing the same line, the changes can be automatically merged\textbf{*}
		\end{itemize}

		\item Easy to see differences between versions of files
		\item Safely remove old bits of code, and have a backup
	\end{itemize}
\end{frame}

\begin{frame}
	\frametitle{So what is Git?}
	\begin{itemize}
		\item Git is a \footnotedef{decentralised}{without a central server to check code in and out of} version control system
		\item It was created by \textbf{Linus Torvalds} as a replacement for the previous VCS, Bitkeeper.
		\begin{itemize}
			\item Four days to initial release
			\item Initial performance goals within a month
			\item In use in two and a half months
			\item Four months until feature completion, so handed off
		\end{itemize}

	\end{itemize}
\end{frame}

	\begin{frame}
		\frametitle{How Does it Work?}

		\begin{itemize}
			\item Git stores all your files in a \footnotedef{repository}{a repository is an on-disk data structure which stores metadata for a set of files and/or directory structure\cite{WikiRepo}}
			\item Git stores snapshots of files
		\end{itemize}
		\begin{center}
			\includegraphics[width=0.8\textwidth]{images/snapshots}
		\end{center}
	\end{frame}


\subsection{What Does Using Git Look Like?}

	\begin{frame}
		\frametitle{Example Git History}
		\label{git_history_examples}

		Let's see some examples of good Git history from other repositories:
		\begin{itemize}
			\item \href{https://github.com/jamietanna/dotfiles-arch}{My dotfiles}
			\item \href{https://gitlab.com/jamietanna/mdp}{My Mobile Device Programming coursework}
			\item {\color{red}\href{https://git.kernel.org}{The Linux Kernel}}
				\begin{itemize}
					\item i.e. \href{https://git.kernel.org/cgit/linux/kernel/git/tip/tip.git/log/}{\texttt{tip.git}}
				\end{itemize}

			\item \href{https://github.com/spring-projects/spring-boot}{\texttt{spring-boot}}
		\end{itemize}
	\end{frame}

	\begin{frame}
		\frametitle{Getting Started}
		\begin{exampleblock}{First Steps with Git}
			In order to first play with Git, we have a couple of configuration changes to make:
			\begin{itemize}
 				\item \texttt{git config --global user.name "Jamie Tanna"}
 				\item \texttt{git config --global user.email jamie@jamietanna.co.uk}
			\end{itemize}
		\end{exampleblock}
	\end{frame}

	\begin{frame}
		\frametitle{Getting Started}
		\begin{exampleblock}{Creating a Repository}
			We need to first create our repository:
			\begin{itemize}
 				\item \texttt{cd /path/to/folder}
 				\item \texttt{git init}
			\end{itemize}
		\end{exampleblock}
	\end{frame}

	\begin{frame}
		\frametitle{Making Some Changes}
		\begin{exampleblock}{Create some files}
			\begin{itemize}
				\item \texttt{\$EDITOR helloworld.py}
				\item \texttt{\$EDITOR README.md}
				\item \texttt{git status}
				\item \footnotedef{Stage}{Add files to be \textbf{ready to be} committed} the files: \texttt{git add helloworld.py README.py}
				\item \texttt{git status}
			\end{itemize}
		\end{exampleblock}
		\begin{center}
			\includegraphics[width=0.6\textwidth]{images/areas.png}
		\end{center}
	\end{frame}

	\begin{frame}
		\frametitle{Saving Our Changes}
		\begin{exampleblock}{Committing}
			\begin{itemize}
				\item \texttt{git status}
				\item \texttt{git commit}\footnote{We'll cover writing commit messages later, on slide \hyperlink{sub:good_commit_messages}{\ref{sub:good_commit_messages}}}
				\item \texttt{git status}
			\end{itemize}
		\end{exampleblock}
	\end{frame}

	\begin{frame}
		\frametitle{Seeing What's Changed}
		\begin{exampleblock}{Making a few more changes}
			\begin{itemize}
				\item \texttt{\$EDITOR helloworld.py}
				\item \texttt{git status}
				\item Seeing what's changed: \texttt{git diff}
				\item \texttt{git add helloworld.py}
				\item \texttt{git diff} vs \texttt{git diff --cached}?
			\end{itemize}
		\end{exampleblock}
	\end{frame}

	\begin{frame}
		\frametitle{Unstaging Files}
		\begin{exampleblock}{``Oops, I didn't mean to add that file''}
			\begin{itemize}
				\item \texttt{\$EDITOR helloworld.py}
				\item \texttt{\$EDITOR README.md}
				\item \texttt{git status}
				\item \texttt{git add helloworld.py}
				\item \texttt{git reset}
			\end{itemize}
		\end{exampleblock}
	\end{frame}

	\begin{frame}
		\frametitle{Viewing History}
		\begin{exampleblock}{How can I see what I've done?}
			\begin{itemize}
				\item \texttt{git log}
				\item \texttt{git log --oneline}
				\item \texttt{git show \$SHA}
			\end{itemize}
		\end{exampleblock}
	\end{frame}

% Branching
	\begin{frame}
		\frametitle{Branching and Merging}
		\begin{exampleblock}{Branching and Merging}
			\begin{itemize}
				\item Imagine our current tree like so:
			\end{itemize}
		\end{exampleblock}

		\begin{center}
			\includegraphics[width=0.7\textwidth]{./images/basic-branching-1.png}
		\end{center}
	\end{frame}

	\begin{frame}
		\frametitle{Branching and Merging}
		\begin{exampleblock}{Creating a Branch}
			\begin{itemize}
				\item \texttt{git branch}
				\item \texttt{git branch iss53}
				\item \texttt{git branch}
				\item \texttt{git checkout iss53}
				\item or: \texttt{git checkout -b iss53}
			\end{itemize}
		\end{exampleblock}

		\begin{center}
			\includegraphics[width=0.7\textwidth]{./images/basic-branching-2.png}
		\end{center}
	\end{frame}

	\begin{frame}
		\frametitle{Branching and Merging}
		\begin{exampleblock}{Making a Change on a Branch}
			\begin{itemize}
				\item Now you need to make a fix on \texttt{iss53}:
				\item \texttt{\$EDITOR \$FILENAME}
				\item \texttt{git add \$FILENAME}
				\item \texttt{git commit}
			\end{itemize}
		\end{exampleblock}
		\begin{center}
			\includegraphics[width=0.8\textwidth]{./images/basic-branching-3.png}
		\end{center}
	\end{frame}

	\begin{frame}
		\frametitle{Branching and Merging}
		\begin{exampleblock}{Changing Another Branch}
			\begin{itemize}
				\item \texttt{git checkout master}
				\item \texttt{git checkout -b hotfix}
				\item \texttt{\$EDITOR \$FILENAME}
				\item \texttt{git add \$FILENAME}
				\item \texttt{git commit}
				\item \texttt{git log --graph}
			\end{itemize}
		\end{exampleblock}
		\begin{center}
			\includegraphics[width=0.65\textwidth]{./images/basic-branching-4.png}
		\end{center}
	\end{frame}

	\begin{frame}
		\frametitle{Branching and Merging}
		\begin{exampleblock}{Merging Branches}
			\begin{itemize}
				\item Then we need to merge the changes into master
				\item \texttt{git checkout master}
				\item \texttt{git merge hotfix}
			\end{itemize}
		\end{exampleblock}
		\begin{center}
			\includegraphics[width=0.7\textwidth]{./images/basic-branching-5.png}
		\end{center}
	\end{frame}

	\begin{frame}
		\frametitle{Branching and Merging}

		\begin{itemize}
			\item For a bit more in-depth examples of branching and merging, I'd definitely recommend the Git Book\cite{GitBookGitBranching} and the Github online Git tutorial\cite{GithubTryGit}.
		\end{itemize}
	\end{frame}

%</Branching>
	\begin{frame}
		\frametitle{Working With Other Peoples' Code}
		\begin{exampleblock}{Working on other projects}
			\begin{itemize}
				\item \texttt{git clone \$URL}
				\item \texttt{git pull}
				\item \texttt{git remote}\footnote{\textbf{remote}: {A remote is a non-local repository; anything but our current repo}}
			\end{itemize}
		\end{exampleblock}
	\end{frame}

\label{git_best_practices}
\subsection{Git Best Practices}

	\begin{frame}
		\label{sub:actually_using_git}
		\frametitle{Actually Getting Started With Using Git}
		\begin{itemize}
			\item All well and good knowing the commands; how do you get good at it?
			\item Have to embed it into your workflow
			\item Have to use it in every project you work with
			\item Use often, learn more, repeat
		\end{itemize}
	\end{frame}

	\begin{frame}
		\label{sub:good_commit_messages}
		\frametitle{Writing Good Commit Messages}
		\begin{figure}
			\centering
			\includegraphics[width=0.9\textwidth]{./images/git_commit.png}
			\caption{Alt-text: \texttt{Merge branch `asdfasjkfdlas/alkdjf' into sdkjfls-final}~\cite{XKCDGitCommit}}
		\end{figure}
	\end{frame}

	\begin{frame}
		\frametitle{Branches}

		\begin{itemize}
			\item Keep master stable; only working features
				\begin{itemize}
					\item Why?
				\end{itemize}
			\item Develop in a separate branch i.e. \texttt{login-system}, merge once complete
		\end{itemize}

	\end{frame}

	\begin{frame}
		\frametitle{Ignoring Files}

		\begin{itemize}
			\item We don't want to track all the files in our directory
				\begin{itemize}
						\item \texttt{.pyc}, \texttt{.pdf}, \texttt{.o}
						\item Secret config variables
						\item Intermediate build files
						\item Compiled binaries
				\end{itemize}
			\item \url{http://git-scm.com/docs/gitignore}
			\item Example gitignores can be found at \url{https://github.com/github/gitignore/tree/master/Global}
		\end{itemize}

	\end{frame}



\begin{frame}[fragile]
	\frametitle{Ignoring Files}
	\begin{itemize}
			\item We want to store a global gitignore, for instance:
	\end{itemize}

	\begin{lstlisting}[caption=\texttt{.global\_gitignore}]
	[._]*.s[a-w][a-z]
	[._]s[a-w][a-z]
	*.un~
	Session.vim
	.netrwhist
	*~
	\end{lstlisting}

	\begin{itemize}
		\item And then run \texttt{git config --global core.excludesfile \textasciitilde/.gitignore\_global}
	\end{itemize}

\end{frame}

\begin{frame}[fragile]
	\frametitle{Ignoring Files}
	\begin{itemize}
			\item Useful for enforcing consistency over all contributors in a project
			\item We can also store them per-repository:
	\end{itemize}

	\begin{lstlisting}[caption=\texttt{G53CMP.gitignore}]
	*.o
	*.hi
	Parser.hs
	TAMCodeParser.hs
	hmtc
	\end{lstlisting}
\end{frame}


	\begin{frame}
		\frametitle{Writing Good Commit Messages}

		\begin{itemize}
			\item Quickly skip back to slide \hyperlink{git_history_examples}{\ref{git_history_examples}} to look at examples of Git history
			\item 50 chars -- summary
			\item 72 chars wrapped -- explanation
			\begin{itemize}
				\item Why have you added this change?
				\item What benefit/effect does it add to the repo?
				\item Does it solve any bugs/issues?
				\item Does it solve any bugs/issues?
				\item Explain \textbf{what and why}, not how
			\end{itemize}

			\item Bullet points
			\item Code snippets
			\item Adapted from~\cite{ChrisBeamsGitCommit} and~\cite{GHErlangOTPWikiWGCM}
		\end{itemize}
	\end{frame}


\label{hosted_vcs_compare}
\subsection{Github and Other Git Hosting}

\begin{frame}
	\frametitle{Why Would I Want Hosted Git Repos?}
	\begin{itemize}
		\item Why would you want to host it?
		\item Didn't I say centralised was a bad idea though?
		\item Backup against device crash, inaccessibility
		\item Collaboration
		\item Convenience
		\item Continuous Integration
	\end{itemize}
\end{frame}

\begin{frame}
	\frametitle{GitLab}
	\AddToShipoutPictureFG*{\includegraphics[width=\paperwidth]{backgrounds/gitlab.pdf}}

	\begin{itemize}
		\item \url{https://www.gitlab.com}
		\item Open Source project -- able to run it yourself
		\item Hosted in the department -- \url{http://git.cs.nott.ac.uk}
		\item My new hosting for new projects
		\item Integrated Continuous Integration
		\item Awesome platform, with some great community work and feedback (due to it being OSS)
		\item Unlimited public + private repos (forever)
	\end{itemize}
\end{frame}

\begin{frame}
	\AddToShipoutPictureFG*{\includegraphics[width=\paperwidth]{backgrounds/bitbucket.pdf}}
	\frametitle{BitBucket}

	\begin{itemize}
		\item \url{https://bitbucket.org}
		\item Proprietary software
		\item Previously useful for private repos, now would recommend the use of GitLab
		\item Unlimited public + private repos
	\end{itemize}
\end{frame}

\begin{frame}
	\AddToShipoutPictureFG*{\includegraphics[width=\paperwidth]{backgrounds/github.pdf}}
	\frametitle{Github}

	\begin{itemize}
		\item \url{https://www.github.com}
		\item Proprietary software
		\item Github is \emph{one of} the most popular repository hosting sites, losing ground to Gitlab
		\item External Continuous Integration
		\item Unlimited public, paid private repos
	\end{itemize}
\end{frame}


\section{Hacktoberfest Workshop}

\begin{frame}{Overview}
	\begin{itemize}
		\item Getting started with how to use Git
		\item Hacktoberfest
		\item Making your first Open Source contribution
	\end{itemize}
\end{frame}

\subsection{Hacktoberfest}
\label{sub:Hacktoberfest}
% {{{
\begin{frame}{Hacktoberfest}
	\begin{block}{More details}
		\begin{itemize}
			\item See \url{https://jvt.me/hacktoberfest/opensource/freesoftware/2016/09/30/hacktoberfest.html} for more in-depth discussions than listed on the slides
		\end{itemize}
	\end{block}
\end{frame}

\begin{frame}{Hacktoberfest}
	\begin{itemize}
		\item What is Hacktoberfest?
		\item What do you have to do?
		\item What's the end goal?
		\item What's the reward?
	\end{itemize}
\end{frame}

\begin{frame}{Finding a Project}
	\begin{itemize}
		\item Find something you use regularly
		\item HackSoc repos
		\item \texttt{Hacktoberfest} label projects
	\end{itemize}
\end{frame}

\begin{frame}{Contributing}
	\begin{itemize}
		\item Getting the environment set up
		\item Making the contribution
		\item Checking things still work
		\item Ensure project standards are being followed:
			\begin{itemize}
				\item \texttt{CONTRIBUTING}
				\item \texttt{LICENSE}
				\item Code
				\item Commits
				\item Pull Request
			\end{itemize}
		\item Committing and pushing
		\item Sending a Pull Request
	\end{itemize}
\end{frame}

\begin{frame}{Making Your First Contribution}
	\begin{itemize}
		\item Let's contribute to Hacksoc's website!
		\item Original Pull Request: \url{https://github.com/HackSocNotts/HackSocNotts.github.io/pulls/11}
	\end{itemize}
\end{frame}
% }}}

\section{Conclusion}

\begin{frame}{Summary}

	Get the source of this presentation at

	\begin{center}\url{https://github.com/jamietanna/gittalk15}\end{center}

	For further reading, check out the incredibly awesome (and free) \href{http://git-scm.com/book/en/v2}{Git Book}.

	This presentation is licensed under a
	\href{http://creativecommons.org/licenses/by-sa/4.0/}{Creative Commons
	Attribution-ShareAlike 4.0 International License}.

	\begin{center}\ccbysa\end{center}

\end{frame}

% \plain{Questions?}

\begin{frame}[allowframebreaks]

	\frametitle{References}

	\nocite{*}
	\bibliography{./intro-to-git-workshop.bib}
	\bibliographystyle{abbrv}

\end{frame}

\end{document}
